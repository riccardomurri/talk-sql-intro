\PassOptionsToPackage{table,usenames,dvipsnames,x11names}{xcolor}
\documentclass[english,serif,mathserif]{beamer}
\usetheme[informal]{gc3}

\usepackage[T1]{fontenc}
\usepackage[utf8]{inputenc}
\usepackage{babel}

\usepackage{gc3}

\usepackage{colortbl}
\makeatletter
\rowcolors{1}{uzh@blue!10}{white}
\makeatother

\usepackage{dcolumn}
\newcolumntype{d}[1]{D{.}{\cdot}{#1} }


\begin{document}

%% Optional Argument in [Brackets]: Short Title for Footline
\title[Databases]{A Short Introduction to Relational Databases}
\subtitle{Part 1}
\author{Riccardo Murri \texttt{<riccardo.murri@uzh.ch>}}
\date{2019-04-09}

%% Makes the title slide
\maketitle


\begin{frame}
  \centering

  \+
  {\Huge\itshape Databases}
\end{frame}

\begin{frame}
  \centering

  ``A \textbf{database} is an organized collection of data''

  \uncover<2>{
    \+
    ``A \textbf{database management system} (DBMS) is a computer software application
    \emph{[...]} A general-purpose DBMS is designed to allow the
    definition, creation, querying, update, and administration of databases.''
  }

  \+
  \begin{flushright}
    --- \url{https://en.wikipedia.org/wiki/Database}
  \end{flushright}
\end{frame}


\begin{frame}
  \frametitle{\emph{Relational} DBs}

  All major general purpose DBMS's are based on the so-called relational
  data model.

  \+ This means that all data is stored in a number of tables (with named
  columns), such as:

  \begin{center}
    \small
    \begin{tabular}{lll}
      \textbf{usr}
             & \textbf{size}
                           & \textbf{path}                  \\
      \hline
      usr264 &          17 & /scratch/iftp/usr264/cp1.log   \\
      usr116 & 19362662400 & /scratch/id/usr116/vkeller.tar \\
      usr116 &     3379200 & /scratch/id/usr116/test.tar    \\
      usr264 &          16 & /scratch/iftp/usr264/cp2.log   \\
      usr345 &      877366 & /scratch/aim/usr345/bwa/bwa    \\
    \end{tabular}
  \end{center}

  For historical and mathematical reasons such tables are referred to as
  \emph{relations.}
\end{frame}


\begin{frame}
  \frametitle{How are relational DBs programmed?}
  The success of relational databases is largely due to the existence of
  powerful programming languages for writing database queries.

  \+ The most important such language is SQL (“Structured Query Language”,
  sometimes pronounced “sequel”).

  % \+ Important properties:
  % \begin{itemize}
  % \item \textbf{convenient}: queries can be written with little effort
  % \item \textbf{efficient}: even for large data sets, a good DBMS can answer
  %   queries written in SQL quickly (compared to the fastest possible)
  % \end{itemize}
\end{frame}


\part{Relational data model}
\begin{frame}[fragile]
  \frametitle{Relational data  model}

  \normalsize
  A \textbf{relational database} is a set of relations.

  \+
  A \textbf{relation} is a set of tuples.

  \+
  A relation can be represented by listing all groups of related elements
  --- the result is a ``table''.

  \+
  \small
  \begin{center}
    \begin{tabular}{lll}
      \textbf{usr}
             & \textbf{size}
                           & \textbf{path}                  \\
      \hline
      usr264 &          17 & /scratch/iftp/usr264/cp1.log   \\
      usr116 & 19362662400 & /scratch/id/usr116/vkeller.tar \\
      usr116 &     3379200 & /scratch/id/usr116/test.tar    \\
      usr264 &          16 & /scratch/iftp/usr264/cp2.log   \\
      usr345 &      877366 & /scratch/aim/usr345/bwa/bwa    \\
    \end{tabular}
  \end{center}

  \+
  \smaller
  Much of the material in these slides comes originally from:
  {\em R.~Pagh, Introduction to Database Systems,
    \url{http://www.itu.dk/people/pagh/IDB05/}}
\end{frame}


\begin{frame}[fragile]
  \frametitle{What is a tuple?}
  \small

  A relation consists of so-called tuples.

  \+
  A tuple is an \emph{ordered} list of values.

  \+
  Tuples are usually written in parentheses, with commas separating the
  values (or components), e.g.:

  \begin{quotation}
    \texttt{(Star Wars, 1977, 124, color)}
  \end{quotation}
  which contains the four values \texttt{Star Wars}, \texttt{1977},
  \texttt{124}, and \texttt{color}.

  \+
  \emph{Note:} Order is significant, e.g., the tuple \texttt{(Star Wars, 124, 1977,
    color)} is different from the tuple above.
\end{frame}


\begin{frame}
  \frametitle{What is a relation?}
  \small

  A \textbf{relation} is a \emph{set} of tuples.

  \+
  A relation is always defined on certain sets (called \emph{domains}):
  “the sets from which the values in the tuples come”.

  \+
  \emph{Example:} The tuple \texttt{(Star Wars, 1977, 124, color)} could be
  part of a relation defined on the sets:
  \begin{itemize}
  \item All text strings,
  \item the integers,
  \item the integers, and
  \item \{ \texttt{'color'}, \texttt{'black and white'} \}.
  \end{itemize}
\end{frame}

\begin{frame}
  \frametitle{However\ldots}
  \small

  There are multiple domain sets to which tuple elements could belong.

  \+
  \uncover<1-2>{%
    The tuple \texttt{(Star Wars, 1977, 124, color)} could be
    part of a relation defined on the sets:
    \begin{columns}[t]
      \begin{column}{0.5\linewidth}
        \begin{itemize}
        \item All text strings,
        \item the integers,
        \item \alert<1>{the integers},
        \item \alert<2>{\{ \texttt{'color'}, \texttt{'black~and~white'} \}}.
        \end{itemize}
      \end{column}
      \begin{column}{0.5\linewidth}
        \begin{itemize}
        \item All text strings,
        \item the integers,
        \item \alert<1>{floating-point numbers},
        \item \alert<2>{all text strings}.
        \end{itemize}
      \end{column}
    \end{columns}
  }%

  \+
  \uncover<3>{%
    So the domains, over which the components of tuples (belonging
    to a given relation) are allowed to vary, must be \emph{given}.
    They are part of a relation's definition.
  }%
\end{frame}


\begin{frame}
  \frametitle{Attributes}
  \small

  In order to be able to refer to the different components in a tuple, we will
  assign them names (called \textbf{attributes}).

  \+ \emph{Example:} For the tuple \texttt{(Star Wars, 1977, 124, color)} we
  might choose the attributes \texttt{title}, \texttt{year}, \texttt{length},
  and \texttt{filmType}.

  \+ Attribute \texttt{length} would then refer to the third value of the tuple,
  \texttt{124}.
\end{frame}


\begin{frame}
  \frametitle{How do we write relations?}

  Relations are usually written as two-dimensional tables, with the attributes
  as a first, special row, and the tuples in the remaining rows.

  \+
  \small
  \begin{center}
    \begin{tabular}{lll}
      \textbf{usr}
             & \textbf{size}
                           & \textbf{path}                  \\
      \hline
      usr264 &          17 & /scratch/iftp/usr264/cp1.log   \\
      usr116 & 19362662400 & /scratch/id/usr116/vkeller.tar \\
      usr116 &     3379200 & /scratch/id/usr116/test.tar    \\
      usr264 &          16 & /scratch/iftp/usr264/cp2.log   \\
      usr345 &      877366 & /scratch/aim/usr345/bwa/bwa    \\
    \end{tabular}
  \end{center}
\end{frame}


\begin{frame}
  \frametitle{Equivalent ways of writing relations}
  The order of the rows does not matter \\ (just the \emph{set} of rows).

  \+
  \alert<2>{%
    We may freely reorder the columns, \\
    \emph{provided we reorder attributes accordingly.}
  }%

  \+
  \uncover<2>{%
    \smaller
    (This effectively makes tuples a function from the set of attributes into
    the domains.  Defining tuples as ordered list of values was just a
    preliminary step to get here.)
  }%
\end{frame}


\begin{frame}
  \frametitle{Recap: A formal definition}
  \small

  Let $A$ be a finite set.  It is the set of \emph{attributes}.

  \+
  Let $D: A \to \mathbf{Set}$ be a map from $A$ into the set of (finite) sets,
  i.e., for every $a \in A$ we are given a finite set $D_a$ the \emph{domain} of
  attribute $a$.  $D$ is the relation \emph{schema}.

  \+
  A \emph{tuple} is an element of the product set $\Pi_{a \in A}D_a$, i.e., the choice,
  for every $a \in A$, of an element $t(a) \in D_a$.

  \+
  A \emph{relation} is a set of tuples, i.e., a subset of $\Pi_{a \in A}D_a$.

  \+
  A (relational) \emph{database} is a set of relations.
\end{frame}


\part{Basic SQL}

\begin{frame}
  \frametitle{Naming of parts}

  \centering
  \begin{tabular}{rcl}
    \textbf{Rel. Algebra} &$\Leftrightarrow$& \textbf{SQL DBMS} \\
    \hline
    attributes            && fields, column names \\
    domain                && column type \\
    relation              && table, view${}^{\,*}$
    \\
    tuple                 && row, record \\
  \end{tabular}

  \+
  \smaller
  \raggedright
  \begin{description}
  \item[*] {%
    There is a difference between ``table'' and ``view'' in SQL,
    to be explained later.}
  \end{description}
\end{frame}


\begin{frame}[fragile]
  \frametitle{The SQL \texttt{SELECT} statement}
  \smaller

  \begin{columns}
    \begin{column}{0.5\linewidth}
\begin{sql}
db=> SELECT 'hello, world!';
   ?column?
~-{}-{}-{}-{}-{}-{}-{}-{}-{}-{}-{}-{}-{}-{}-{}~
 hello, world!
(1 row)
\end{sql}
    \end{column}
    \begin{column}{0.5\linewidth}
      The \texttt{SELECT} statement is used to compute expressions in SQL.

      \+ \alert<2>{The result of evaluating a \texttt{SELECT} statement is again a
      \emph{relation} (table).}

      \+
      \only<2>{\textbf{Why is this important?}}
      \only<3>{%
        \alert<3>{%
          It allows \emph{composition} of queries into larger expressions.
        }%
      }%
    \end{column}
  \end{columns}
\end{frame}


\begin{frame}
  \frametitle{What is an algebra?}
  An algebra consists of a set of \textbf{atomic operands}, \\
  and a set of \textbf{operators}.

  \+
  We can form algebraic expressions by applying operators to operands
  \alert{(which can be atomic or expressions themselves)}.

  \+
  For this to work, the result of applying an operator to an expression must be
  again an operand.  In other words, an algebra must be \emph{closed} under all
  given operations.
\end{frame}


\begin{frame}
  \frametitle{What is \emph{relational} algebra?}

  Relational algebra, defined in its basic form by E.~F.~Codd in 1970, has:
  \begin{itemize}
  \item relations as atomic operands, and
  \item various operations on relations as operators (which will be detailed shortly).
  \end{itemize}

  \+
  Relational algebra is the basis of SQL.
\end{frame}


\begin{frame}[fragile]
  \frametitle{Basic SQL syntax}
  \smaller

  \begin{columns}
    \begin{column}{0.5\linewidth}
\begin{sql}
db=> SELECT ~\HL{'hello, world!'}~;
   ?column?
~-{}-{}-{}-{}-{}-{}-{}-{}-{}-{}-{}-{}-{}-{}-{}~
 hello, world!
(1 row)
\end{sql}
    \end{column}
    \begin{column}{0.5\linewidth}
    Strings in SQL are delimited by \emph{single} quotes.
    \end{column}
  \end{columns}
\end{frame}


\begin{frame}[fragile]
  \frametitle{The SQL \texttt{SELECT} statement}
  \smaller

  \begin{columns}
    \begin{column}{0.5\linewidth}
\begin{sql}
db=> SELECT 'hello, world!';
   ~\HL{?column?}~
~-{}-{}-{}-{}-{}-{}-{}-{}-{}-{}-{}-{}-{}-{}-{}~
 hello, world!
(1 row)
\end{sql}
    \end{column}
    \begin{column}{0.5\linewidth}
    Columns in a table must have a name, so the DBMS makes up one.
    \end{column}
  \end{columns}
\end{frame}


\begin{frame}[fragile]
  \frametitle{The SQL \texttt{SELECT} statement}
  \smaller

  \begin{columns}
    \begin{column}{0.5\linewidth}
\begin{sql}
db=> SELECT 'hello, world!'
db->        AS text  ;
   ~\HL{text}~
~-{}-{}-{}-{}-{}-{}-{}-{}-{}-{}-{}-{}-{}-{}-{}~
 hello, world!
(1 row)
\end{sql}
    \end{column}
    \begin{column}{0.5\linewidth}
    A column name can be given with the \texttt{AS} clause.
    \end{column}
  \end{columns}
\end{frame}


\begin{frame}[fragile]
  \frametitle{The SQL \texttt{SELECT} statement}
  \smaller

  \begin{columns}
    \begin{column}{0.5\linewidth}
\begin{sql}
db=> SELECT 'foo'~\HL{,}~ 'bar';
 ?column? | ?column?
~-{}-{}-{}-{}-{}-{}-{}-{}-{}-{}+{}-{}-{}-{}-{}-{}-{}-{}-{}-{}-{}~
 foo      | bar
(1 row)
\end{sql}
    \end{column}
    \begin{column}{0.5\linewidth}
      Multiple expressions can be evaluated by separating them with a comma.

      \+ They define different columns in the result table.
    \end{column}
  \end{columns}
\end{frame}


\begin{frame}
  \frametitle{\texttt{NULL} and \texttt{UNKNOWN}}
  \texttt{NULL} is a special value that may be used for any data type when no other
  value is applicable.

  \+
  Evaluating expressions involving \texttt{NULL}:
  \begin{itemize}
  \item Arithmetic expressions involving \texttt{NULL} evaluate to
    \texttt{NULL}.
  \item Boolean expressions evaluate to \texttt{NULL} only when the
    correct value cannot be deduced.
  \end{itemize}

  \+ Some RDBMSs support a special null value for booleans called
  \texttt{UNKNOWN}.
\end{frame}


\begin{frame}[fragile]
  \frametitle{Selection}

  \texttt{SELECT ... FROM ... WHERE} returns a relation of all rows in a table
  that satisfy a certain predicate.

  \+
\begin{sql}[basicstyle=\ttfamily\tiny]
db=> SELECT *
db-> FROM lustre
db-> WHERE usr='usr116';
  usr   |~\ldots~|   ~\tt size~    | path
~-{}-{}-{}-{}-{}-{}-{}-{}+{}-{}-{}-{}+{}-{}-{}-{}-{}-{}-{}-{}-{}-{}-{}-{}+{}-{}-{}-{}-{}-{}-{}-{}-{}-{}-{}-{}-{}-{}-{}-{}-{}-{}-{}-{}-{}-{}-{}-{}-{}-{}-{}-{}-{}-{}-{}-{}-{}-{}-{}-{}-{}-{}-{}-{}-{}-{}-{}-{}-{}-{}-{}-{}-{}-{}-{}-{}-{}-{}-{}-{}-{}-{}-{}-{}-{}-{}-{}-{}-{}-{}-{}-{}~
 usr116 |~\ldots~| 256958166 | /scratch/id/usr116/pic_babo.tgz
 usr116 |~\ldots~|   2101760 | /scratch/id/usr116/testib.tar
 ~{\em [\ldots]}~
\end{sql}
\end{frame}


\begin{frame}[fragile]
  \frametitle{Selection}

  \texttt{SELECT ... FROM ... WHERE} returns a relation of all rows in a table
  that satisfy a certain predicate.

  \+
\begin{sql}[basicstyle=\ttfamily\tiny]
db=> SELECT ~\HL{*}~
db-> FROM lustre
db-> WHERE usr='usr116';
  usr   |~\ldots~|   ~\tt size~    | path
~-{}-{}-{}-{}-{}-{}-{}-{}+{}-{}-{}-{}+{}-{}-{}-{}-{}-{}-{}-{}-{}-{}-{}-{}+{}-{}-{}-{}-{}-{}-{}-{}-{}-{}-{}-{}-{}-{}-{}-{}-{}-{}-{}-{}-{}-{}-{}-{}-{}-{}-{}-{}-{}-{}-{}-{}-{}-{}-{}-{}-{}-{}-{}-{}-{}-{}-{}-{}-{}-{}-{}-{}-{}-{}-{}-{}-{}-{}-{}-{}-{}-{}-{}-{}-{}-{}-{}-{}-{}-{}-{}-{}~
 usr116 |~\ldots~| 256958166 | /scratch/id/usr116/pic_babo.tgz
 usr116 |~\ldots~|   2101760 | /scratch/id/usr116/testib.tar
 ~{\em [\ldots]}~
\end{sql}

  \texttt{*} is a shorthand for ``all column names''.
\end{frame}


\begin{frame}[fragile]
  \frametitle{Selection}

  \texttt{SELECT ... FROM ... WHERE} returns a relation of all rows in a table
  that satisfy a certain predicate.

  \+
\begin{sql}[basicstyle=\ttfamily\tiny]
db=> SELECT *
db-> FROM ~\HL{lustre}~
db-> WHERE usr='usr116';
  usr   |~\ldots~|   ~\tt size~    | path
~-{}-{}-{}-{}-{}-{}-{}-{}+{}-{}-{}-{}+{}-{}-{}-{}-{}-{}-{}-{}-{}-{}-{}-{}+{}-{}-{}-{}-{}-{}-{}-{}-{}-{}-{}-{}-{}-{}-{}-{}-{}-{}-{}-{}-{}-{}-{}-{}-{}-{}-{}-{}-{}-{}-{}-{}-{}-{}-{}-{}-{}-{}-{}-{}-{}-{}-{}-{}-{}-{}-{}-{}-{}-{}-{}-{}-{}-{}-{}-{}-{}-{}-{}-{}-{}-{}-{}-{}-{}-{}-{}-{}~
 usr116 |~\ldots~| 256958166 | /scratch/id/usr116/pic_babo.tgz
 usr116 |~\ldots~|   2101760 | /scratch/id/usr116/testib.tar
 ~{\em [\ldots]}~
\end{sql}

\texttt{FROM} indicates the table from which to extract rows.

\+ Each table in a SQL database is given a name (at creation time).
\end{frame}


\begin{frame}[fragile]
  \frametitle{Selection}

  \texttt{SELECT ... FROM ... WHERE} returns a relation of all rows in a table
  that satisfy a certain predicate.

  \+
\begin{sql}[basicstyle=\ttfamily\tiny]
db=> SELECT *
db-> FROM lustre
db-> WHERE ~\HL{usr='usr116'}~;
  usr   |~\ldots~|   ~\tt size~    | path
~-{}-{}-{}-{}-{}-{}-{}-{}+{}-{}-{}-{}+{}-{}-{}-{}-{}-{}-{}-{}-{}-{}-{}-{}+{}-{}-{}-{}-{}-{}-{}-{}-{}-{}-{}-{}-{}-{}-{}-{}-{}-{}-{}-{}-{}-{}-{}-{}-{}-{}-{}-{}-{}-{}-{}-{}-{}-{}-{}-{}-{}-{}-{}-{}-{}-{}-{}-{}-{}-{}-{}-{}-{}-{}-{}-{}-{}-{}-{}-{}-{}-{}-{}-{}-{}-{}-{}-{}-{}-{}-{}-{}~
 usr116 |~\ldots~| 256958166 | /scratch/id/usr116/pic_babo.tgz
 usr116 |~\ldots~|   2101760 | /scratch/id/usr116/testib.tar
 ~{\em [\ldots]}~
\end{sql}

The \texttt{WHERE} clause specifies the predicate that selected rows must
satisfy.
\end{frame}


\begin{frame}[fragile]
  \frametitle{Selection}

  \texttt{SELECT ... FROM ... WHERE} returns a relation of all rows in a table
  that satisfy a certain predicate

  \+
\begin{sql}[basicstyle=\ttfamily\tiny]
db=> SELECT *
db-> FROM lustre
db-> WHERE ~\HL{usr}~='usr116';
  usr   |~\ldots~|   ~\tt size~    | path
~-{}-{}-{}-{}-{}-{}-{}-{}+{}-{}-{}-{}+{}-{}-{}-{}-{}-{}-{}-{}-{}-{}-{}-{}+{}-{}-{}-{}-{}-{}-{}-{}-{}-{}-{}-{}-{}-{}-{}-{}-{}-{}-{}-{}-{}-{}-{}-{}-{}-{}-{}-{}-{}-{}-{}-{}-{}-{}-{}-{}-{}-{}-{}-{}-{}-{}-{}-{}-{}-{}-{}-{}-{}-{}-{}-{}-{}-{}-{}-{}-{}-{}-{}-{}-{}-{}-{}-{}-{}-{}-{}-{}~
 usr116 |~\ldots~| 256958166 | /scratch/id/usr116/pic_babo.tgz
 usr116 |~\ldots~|   2101760 | /scratch/id/usr116/testib.tar
 ~{\em [\ldots]}~
\end{sql}

The \texttt{WHERE} clause specifies the predicate that selected rows must
satisfy.

\+ Column names are used in predicate expression to denote the value of the
corresponding attribute value in a tuple.
\end{frame}


\begin{frame}[fragile]
  \frametitle{Selection}

  \texttt{SELECT ... FROM ... WHERE} returns a relation of all rows in a table
  that satisfy a certain predicate

  \+
\begin{sql}[basicstyle=\ttfamily\tiny]
db=> SELECT *
db-> FROM lustre
db-> WHERE usr~\HL{=}~'usr116';
  usr   |~\ldots~|   ~\tt size~    | path
~-{}-{}-{}-{}-{}-{}-{}-{}+{}-{}-{}-{}+{}-{}-{}-{}-{}-{}-{}-{}-{}-{}-{}-{}+{}-{}-{}-{}-{}-{}-{}-{}-{}-{}-{}-{}-{}-{}-{}-{}-{}-{}-{}-{}-{}-{}-{}-{}-{}-{}-{}-{}-{}-{}-{}-{}-{}-{}-{}-{}-{}-{}-{}-{}-{}-{}-{}-{}-{}-{}-{}-{}-{}-{}-{}-{}-{}-{}-{}-{}-{}-{}-{}-{}-{}-{}-{}-{}-{}-{}-{}-{}~
 usr116 |~\ldots~| 256958166 | /scratch/id/usr116/pic_babo.tgz
 usr116 |~\ldots~|   2101760 | /scratch/id/usr116/testib.tar
 ~{\em [\ldots]}~
\end{sql}

The \texttt{WHERE} clause specifies the predicate that selected rows must
satisfy.

\+
If the \texttt{WHERE} clause is omitted, all rows are selected.
\end{frame}


\begin{frame}
  \frametitle{Predicates in \texttt{WHERE} clauses}
  SQL offers a variety of ways of forming conditional expressions in
  \texttt{WHERE} clauses:

  \+
  \begin{itemize}
  \item Comparison operators (used between e.g. a pair of integers or strings):
    \texttt{=}, \texttt{<}, \texttt{<=}, \texttt{>}, \texttt{>=}, \texttt{<>}.
  \item Boolean operators (used to combine conditional expressions):
    \texttt{AND}, \texttt{OR}, \texttt{NOT}.
  \item The \texttt{LIKE} operator used to find strings that match a given
    pattern.
  \item Parentheses can be used to indicate the order of evaluation. If no
    parentheses are present, a standard order is used.
  \end{itemize}
\end{frame}

\begin{frame}[fragile]
  \frametitle{Selection}

  \texttt{SELECT ... FROM ... WHERE} returns a relation of all rows in a table
  that satisfy a certain predicate

  \+
\begin{sql}[basicstyle=\ttfamily\tiny]
db=> SELECT *
db-> FROM lustre
db-> WHERE ~\HL{LENGTH(path)}~<32;
  usr   |~\ldots~|   ~\tt size~    | path
~-{}-{}-{}-{}-{}-{}-{}-{}+{}-{}-{}-{}+{}-{}-{}-{}-{}-{}-{}-{}-{}-{}-{}-{}+{}-{}-{}-{}-{}-{}-{}-{}-{}-{}-{}-{}-{}-{}-{}-{}-{}-{}-{}-{}-{}-{}-{}-{}-{}-{}-{}-{}-{}-{}-{}-{}-{}-{}-{}-{}-{}-{}-{}-{}-{}-{}-{}-{}-{}-{}-{}-{}-{}-{}-{}-{}-{}-{}-{}-{}-{}-{}-{}-{}-{}-{}-{}-{}-{}-{}-{}-{}~
 usr116 |~\ldots~| 256958166 | /scratch/id/usr116/pic_babo.tgz
 usr116 |~\ldots~|   2101760 | /scratch/id/usr116/testib.tar
 ~{\em [\ldots]}~
\end{sql}

The \texttt{WHERE} clause specifies the predicate that selected rows must
satisfy.

\+ Functions can be applied to column values.

\+ RDBMS provide some built-in functions but generally let users also define
their own (UDF).
\end{frame}


% \begin{frame}[fragile]
%   \frametitle{Selection}

%   \texttt{SELECT ... FROM ... WHERE} returns a relation of all rows in a table
%   that satisfy a certain predicate

%   \+
% \begin{sql}[basicstyle=\ttfamily\tiny]
% db=> SELECT *
% db-> FROM lustre
% db-> WHERE usr='usr116';
%   usr   |~\ldots~|   ~\tt size~    | path
% --------+--------+-----------+-------------------------------------------------------------------
%  usr116 |~\ldots~| 256958166 | /scratch/id/usr116/pic_babo.tgz
%  usr116 |~\ldots~|   2101760 | /scratch/id/usr116/testib.tar
%  ~{\em [\ldots]}~
% \end{sql}

% The \texttt{WHERE} clause specifies the predicate that selected rows must
% satisfy.
% \end{frame}


\begin{frame}[fragile]
  \frametitle{Projection}

  A list of column names or expressions can be given to \texttt{SELECT}; the
  resulting relation will contain only the specified columns.

  \+
\begin{sql}[basicstyle=\ttfamily\tiny]
db=> SELECT ~\HL{usr, size, path}~
db-> FROM lustre;
  usr  |  ~\tt size~  |              path
~-{}-{}-{}-{}-{}-{}-{}+{}-{}-{}-{}-{}-{}-{}-{}-{}+{}-{}-{}-{}-{}-{}-{}-{}-{}-{}-{}-{}-{}-{}-{}-{}-{}-{}-{}-{}-{}-{}-{}-{}-{}-{}-{}-{}-{}-{}-{}-{}-{}-{}~
 us345 |    100 | /scratch/aim/bin/run_fastqc.bat
 us345 | 507539 | /scratch/aim/bin/sam-1.32.jar
 ~{\em [\ldots]}~
\end{sql}
\end{frame}


\begin{frame}[fragile]
  \frametitle{\emph{Set} vs. \emph{Bag} semantics}

  A list of column names or expressions can be given to \texttt{SELECT}; the
  resulting relation will contain only the specified columns.

  \+
\begin{sql}[basicstyle=\ttfamily\tiny]
db=> SELECT usr
db-> FROM lustre;
  usr
~-{}-{}-{}-{}-{}-{}-{}~
 ~\HL{usr345}~
 ~\HL{usr345}~
 ~\HL{usr345}~
 ~{\em [\ldots]}~
\end{sql}

  However, \alert{selecting a subset of attributes may lead to \emph{non-unique} tuples.}
\end{frame}


\begin{frame}[fragile]
  \frametitle{\emph{Set} vs. \emph{Bag} semantics}

  SQL defines a relation as a \emph{list} of tuples, \\ i.e., an
  ordered sequence of rows, \\ allowing identical tuples to appear.

  \+ The reason is \emph{efficiency}: eliminating duplicates is
  (computationally) costly, so is only done at the programmers' request.
\end{frame}


\begin{frame}[fragile]
  \frametitle{DISTINCT}

  The \texttt{DISTINCT} modifier is used to force removal of duplicate values in
  an expression:

  \+
\begin{sql}[basicstyle=\ttfamily\tiny]
db=> SELECT DISTINCT usr
db-> FROM lustre;
  usr
~-{}-{}-{}-{}-{}-{}-{}~
 ~\HL{usr345}~
 ~\HL{usr116}~
 ~{\em [\ldots]}~
\end{sql}
\end{frame}


\begin{frame}[fragile]
  \frametitle{Keys}

  A subset $A$ of the attributes of relation $R$ such that there are no
  duplicates when $R$ is projected upon $A$ is called a \textbf{key}.

\begin{sql}
db=> SELECT path FROM lustre;
 ~{\em [\ldots]}~
~\HL{(30356776 rows)}~

db=> SELECT DISTINCT path FROM lustre;
 ~{\em [\ldots]}~
~\HL{(30356776 rows)}~
\end{sql}

\end{frame}

\begin{frame}
  \frametitle{Keys}

  A subset $A$ of the attributes of relation $R$ such that there are no
  duplicates when $R$ is projected upon $A$ is called a \textbf{key}.

  \+
  \alert{The uniqueness constraints of the key must hold however the
  relation is updated and modified.}  \\ In other words, a key constraint
  is an existential attribute of the relation, not just an accident of
  the data.

  \+
  So, key constraints are specified when the table is created.
\end{frame}


\begin{frame}[fragile]
  \frametitle{Ordered relations, again}

  SQL defines a relation as a \emph{list} of tuples, i.e., an ordered
  sequence of rows, allowing identical tuples to appear.

  \+ Can you think of another reason why SQL defines a relation as an
  \emph{ordered} sequence?

  \+
  (Answer on next slide.)
\end{frame}


\begin{frame}[fragile]
  \frametitle{Sorting}
  An additional clause \texttt{ORDER BY} allows sorting on an arbitrary
  expression of the attribute values.

  \+
  \begin{columns}
    \begin{column}{0.5\linewidth}
\begin{sql}
db=> SELECT DISTINCT usr
db-> FROM lustre
db-> ~\HL{ORDER BY}~ usr ASC
db-> LIMIT 3;
  usr
~-{}-{}-{}-{}-{}-{}-{}~
 us293
 us319
 us320
(3 rows)
\end{sql}
    \end{column}
    \begin{column}{0.5\linewidth}

    \end{column}
  \end{columns}
\end{frame}


\begin{frame}[fragile]
  \frametitle{Sorting}
  An additional clause \texttt{ORDER BY} allows sorting on an arbitrary
  expression of the attribute values.

  \+
  \begin{columns}
    \begin{column}{0.5\linewidth}
\begin{sql}
db=> SELECT DISTINCT usr
db-> FROM lustre
db-> ORDER BY ~\HL{usr}~ ASC
db-> LIMIT 3;
  usr
~-{}-{}-{}-{}-{}-{}-{}~
 us293
 us319
 us320
(3 rows)
\end{sql}
    \end{column}
    \begin{column}{0.5\linewidth}
      Any valid expression can be used as a sorting key!
    \end{column}
  \end{columns}
\end{frame}


\begin{frame}[fragile]
  \frametitle{Sorting}
  An additional clause \texttt{ORDER BY} allows sorting on an arbitrary
  expression of the attribute values.

  \+
  \begin{columns}
    \begin{column}{0.5\linewidth}
\begin{sql}
db=> SELECT DISTINCT usr
db-> FROM lustre
db-> ORDER BY usr ~\HL{ASC}~
db-> LIMIT 3;
  usr
~-{}-{}-{}-{}-{}-{}-{}~
 us293
 us319
 us320
(3 rows)
\end{sql}
    \end{column}
    \begin{column}{0.5\linewidth}
      Specify \textbf{ascending} sort order.

      \+ Use \texttt{DESC} for the opposite ordering.
    \end{column}
  \end{columns}
\end{frame}


\begin{frame}[fragile]
  \frametitle{Aside: LIMIT}
  An additional clause \texttt{LIMIT} allows specifying a maximum size
  of the result set.  It can be applied to \emph{any} SQL query.

  \+
  \begin{columns}
    \begin{column}{0.5\linewidth}
\begin{sql}
db=> SELECT DISTINCT usr
db-> FROM lustre
db-> ORDER BY usr ASC
db-> ~\HL{LIMIT 3}~;
  usr
~-{}-{}-{}-{}-{}-{}-{}~
 us293
 us319
 us320
(3 rows)
\end{sql}
    \end{column}
    \begin{column}{0.5\linewidth}

    \end{column}
  \end{columns}
\end{frame}


% \begin{frame}[fragile]
%   \frametitle{Sorting}
%   An additional clause \texttt{ORDER BY} allows sorting on an arbitrary
%   expression of the attribute values.

%   \+
%   \begin{columns}
%     \begin{column}{0.5\linewidth}
% \begin{sql}
% db=> SELECT DISTINCT usr
% db-> FROM lustre
% db-> ORDER BY size ASC
% db-> LIMIT 3;
%   usr
% -------
%  us293
%  us319
%  us320
% (3 rows)
% \end{sql}
%     \end{column}
%     \begin{column}{0.5\linewidth}

%     \end{column}
%   \end{columns}
% \end{frame}


\part{Aggregate functions}

\begin{frame}[fragile]
  \frametitle{Computing on aggregates}

  Some functions by definition operate on a group of rows and return a single
  value out of the group:

  \+
  \begin{sql}
db=> SELECT AVG(size) FROM lustre;
         avg
~-{}-{}-{}-{}-{}-{}-{}-{}-{}-{}-{}-{}-{}-{}-{}-{}-{}-{}-{}-{}-{}-{}~
 4815351.200747734213

db=> SELECT MAX(size) FROM lustre;
     max
~-{}-{}-{}-{}-{}-{}-{}-{}-{}-{}-{}-{}-{}-{}~
  354519221688

db=> SELECT COUNT(*) FROM lustre;
  count
~-{}-{}-{}-{}-{}-{}-{}-{}-{}-{}~
 30356776
\end{sql}
\end{frame}


\begin{frame}[fragile]
  \frametitle{Computing on aggregates}

  By default, aggregation is over all selected rows.

  \+
  A \texttt{GROUP BY} clause can be used to specify how rows should be grouped;
  expressions involving aggregate functions will be computed once per each group.

  \+
  \begin{sql}
 db=> SELECT usr, MAX(size)
 db-> FROM lustre
 db-> GROUP BY usr;
  usr   |     max
~-{}-{}-{}-{}-{}-{}-{}-{}+{}-{}-{}-{}-{}-{}-{}-{}-{}-{}-{}-{}-{}-{}-{}~
 usr25  | 354519221688
 usr324 |   3026168420
 usr234 |   2038075132
 ~{\ldots}~
\end{sql}

\end{frame}


\begin{frame}[fragile]
  \frametitle{Computing on aggregates}

  An additional \texttt{HAVING} clause applies a predicate to further select
  groups based on some expression.

  \+
  For example, compute average file size per user, but only consider users that
  have at least 10000 files on the system:
  \begin{sql}
 db=> SELECT usr, AVG(size)
 db-> FROM lustre
 db-> GROUP BY usr
 db-> ~\HL{HAVING COUNT(path)>10000}~;
  usr   |          avg
~-{}-{}-{}-{}-{}-{}-{}-{}+{}-{}-{}-{}-{}-{}-{}-{}-{}-{}-{}-{}-{}-{}-{}-{}-{}-{}-{}-{}-{}-{}-{}-{}~
 usr25  | 19942289.171742225897
 usr324 |  1891531.025947844280
 usr234 |  1443121.015355730924
 ~{\ldots}~
\end{sql}
\end{frame}


\begin{frame}[fragile]
  \frametitle{Computing on aggregates}

  For example, compute average file size per user, but only consider users that
  have at least 10000 files on the system:
  \begin{sql}
 db=> SELECT usr, AVG(size)
 db-> FROM lustre
 db-> GROUP BY usr
 db-> ~\HL{HAVING COUNT(path)>10000}~;
  usr   |          avg
~-{}-{}-{}-{}-{}-{}-{}-{}+{}-{}-{}-{}-{}-{}-{}-{}-{}-{}-{}-{}-{}-{}-{}-{}-{}-{}-{}-{}-{}-{}-{}-{}~
 usr25  | 19942289.171742225897
 usr324 |  1891531.025947844280
 usr234 |  1443121.015355730924
 ~{\ldots}~
\end{sql}

  \+
  \alert{Could this be done with a \texttt{WHERE} clause?  Why?}
\end{frame}


\part{Subqueries}
\begin{frame}[fragile]
  \frametitle{Subqueries}
  \small

  Until now, we have seen SQL queries of the form:
\begin{semiverbatim}
SELECT \emph{list of attributes}
FROM \emph{list of relations}
WHERE \emph{condition};
\end{semiverbatim}

  \+
  What we haven’t used is that:
  \begin{itemize}
  \item In any place where a relation is allowed, we may put an SQL query (a
    subquery) computing a relation.
  \item In any place where an “atomic value” (e.g. string or integer) is
    allowed, we may put an SQL subquery computing a relation with one value of
    this type (i.e., having one attribute and one tuple).
  \end{itemize}
\end{frame}


\begin{frame}
  \frametitle{Subqueries in the \texttt{FROM} clause}
  Instead of just relations, we may use SQL queries in the \texttt{FROM} clause of
  a \texttt{SELECT} statement.

  \+
  If we need a name for referring to the relation computed by the subquery, an
  additional \texttt{AS \emph{name}} is appended after the subquery.
  This allows us to use values from the subquery results in other expressions.

  \+
  Subqueries should always be surrounded by parentheses.
\end{frame}


\begin{frame}
  \frametitle{Subqueries producing scalar values}

  When a query produces a relation with one attribute and one tuple, it can
  be used in any place where we can put an atomic (or scalar) value.

  \+ In places where an atomic value is expected, SQL regards a relation
  instance containing one atomic value
  $x$ to be the same as the value $x$.

  \+
  If such a subquery does not result in exactly one tuple, it is a run-time
  error, and the SQL query cannot be completed.
\end{frame}


\begin{frame}
  \frametitle{Subqueries in conditions}
  \small

  One common use of subqueries is in the \texttt{WHERE} part of
  a \texttt{SELECT} statememt.

  \+ There are several operators in SQL that apply to a
  relation~$S$, result of a subquery, and produce a boolean result:
  \begin{itemize}
  \item \HL{\texttt{EXISTS} $S$} is true if and only if $S$ is not empty.
  \item \HL{$t$ \texttt{IN} $S$} is true if and only if $t$ is a tuple in $S$.
  \end{itemize}

  \+ If $S$ is unary (has just one attribute):
  \begin{itemize}
  \item \HL{\texttt{$x$ > ALL $S$}} is true if and only if $x$ is greater than all
    values in $S$.
  \item \HL{\texttt{$x$ > ANY $S$}} is true if and only if $x$ is greater
    than some value in $S$
  \end{itemize}
  (and similarly for other comparison operators).
\end{frame}


\part{Appendix}
\begin{frame}
  \frametitle{References}

  Pagh, R.: ``Introduction to Database Systems'',
  \url{http://www.itu.dk/people/pagh/IDB05/}

  \+
  Codd, E.~F.: ``A relational model for large shared data banks'', \emph{Comm.
  ACM} 13:6, pp.~377--387, 1970
\end{frame}


\end{document}


%%% Local Variables:
%%% mode: latex
%%% TeX-master: t
%%% End:
